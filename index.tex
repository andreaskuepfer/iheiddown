% % % % % % % % % % % % %
% Preamble % % %
% % % % % % % % % % % % %
\documentclass{iheid}

% Setup macros -- Please complete yaml in index.Rmd
\newcommand{\mytitle}{A Graduate Institute Thesis\\
Template for R Markdown \texttt{iheidddown}:\\}
\newcommand{\myauthor}{James Hollway}
\newcommand{\myemail}{\href{mailto:james.hollway@graduateinstitute.ch}{\nolinkurl{james.hollway@graduateinstitute.ch}}}
\newcommand{\mydepartment}{IRPS}
\newcommand{\mythesisno}{99999} % provided by PhD secretariat
\newcommand{\mykeywords}{rmarkdown, template, reproducible research,
iheid} % only used for PDF metadata
\newcommand{\mysubject}{IHEID dissertation template in
rmarkdown} % only used for PDF metadata
\newcommand{\mylogo}{Logo_CMYK_Hi.eps} % name of the IHEID logo file

% enable link highlighting to be turned off from YAML
\usepackage[colorlinks=false,pdfpagelabels,hidelinks=true]{hyperref}

% add a "DRAFT" footer to every normal page
\fancyfoot[C]{\emph{Draft -- Please do not cite or circulate.}}

% chose biliography options % % %
\usepackage[backend=biber, % if this doesn't work for you, use bibtex8 as a backend
%style=verbose, % use this if you are in the law department
style=authoryear, % use this if you are in any other department
maxcitenames=3, % how many authors to cite before resorting to ``et al''
maxbibnames=99, % in the bibliography we want them all though
sortcites=true, % this re-sorts citations alphabetically when you cite several at once
hyperref=true, % this adds a link to a citations date that points to the reference
%backref=true, % references state on which page they are cited
abbreviate=true,
url=false, % don't add (lengthy) URL information
doi=false, % don't add (lengthy) DOI information
autolang=hyphen]]{biblatex}
\newcommand*{\bibtitle}{References}

% This makes the bibliography left-aligned (not 'justified') and slightly smaller font.
\renewcommand*{\bibfont}{\raggedright\small}

% Change this to the name of your .bib file (usually exported from a citation manager like Zotero or EndNote).
\addbibresource{mybib.bib}

% Setup different header and footer for title page(s)
\fancypagestyle{firstpages} {
	\fancyhf{} % clear header and footer
	\renewcommand{\headrulewidth}{0pt} % remove header line
	\renewcommand{\footrulewidth}{0pt} % remove footer line
}

% Setup custom title format
\makeatletter
\renewcommand\maketitle
  {\vspace*{3cm} % distance from top of page to title
  \centering{
  \Large\sffamily\textbf{\@title}
  \vspace{1cm}\par % distance to programme statement
  \centering\large\sffamily\textbf{THESIS}\\
  submitted at the Graduate Institute\\
  in fulfillment of the requirements of the\\
  \myprogramme
  \vspace*{0.5cm}\par % distance to author
  \centering\large\sffamily{
	by\\
	\vspace{0.5cm} % distance between 'by' and author name
	\@author
  }
  \bigskip\par
  }
  }
\makeatother

% % % % % % % % % % % % %
% Document Starts Here % % %
% % % % % % % % % % % % %
\begin{document}

%%%%%%%%%%%%%%%%%
% TITLEPAGES
%%%%%%%%%%%%%%%%%
% TITLEPAGE 1
\begin{titlepage}
\hspace*{-1.25cm}
\includegraphics[width=0.4\linewidth]{\mylogo}

\maketitle

\vspace*{1cm}

\sffamily{Thesis No. \mythesisno}

\vspace*{1cm}
\sffamily\textbf{Geneva}\\
\sffamily\textbf{\the\year}

\thispagestyle{firstpages} % this line is needed to set a different header/footer than for the rest of the document
\end{titlepage}

% TITLEPAGE 2 (empty page)
\newpage\null\thispagestyle{empty}\newpage

% TITLEPAGE 3
\begin{titlepage}
\vspace*{5cm}

\centering\Large\sffamily\textbf{\mytitle}

\vspace*{10cm}

\large\raisebox{0.1em}{\textcopyright} \sffamily\textbf{\the\year} \hspace{1em} \sffamily\textbf{\MakeUppercase{\mylastname}}

\thispagestyle{firstpages}
\end{titlepage}

% TITLEPAGE 4
\begin{titlepage}
\centering\sffamily{
INSTITUT DE HAUTES ETUDES INTERNATIONALES ET DU DEVELOPPEMENT\\
GRADUATE INSTITUTE OF INTERNATIONAL AND DEVELOPMENT STUDIES
}

\maketitle

\vspace*{1cm}

\sffamily{Thesis No. \mythesisno}

\vspace*{1cm}
\sffamily\textbf{Geneva}\\
\sffamily\textbf{\the\year}

\thispagestyle{firstpages}
\end{titlepage}

% TITLEPAGE 5
\begin{titlepage}

\textcolor{red}{PLACEHOLDER: REPLACE PAGE WITH A DOCUMENT PROVIDED BY PHD SECRETARIAT}

\thispagestyle{firstpages}
\end{titlepage}

% TITLEPAGE 6
\begin{titlepage}
\centering\sffamily{
INSTITUT DE HAUTES ETUDES INTERNATIONALES ET DU DEVELOPPEMENT\\
GRADUATE INSTITUTE OF INTERNATIONAL AND DEVELOPMENT STUDIES
}

\vspace*{1cm}

\begin{center}
\sffamily\large\textbf{RESUME / ABSTRACT}

\sffamily\normalsize{
	\begin{abstract}
	This \emph{R Markdown} template is for writing an Oxford University
 thesis. The template is built using Yihui Xie's \texttt{bookdown}
 package, with heavy inspiration from Chester Ismay's
 \texttt{thesisdown} and the \texttt{OxThesis} \LaTeX~template (most
 recently adapted by John McManigle).

 This template's sample content include illustrations of how to write a
 thesis in R Markdown, and largely follows the structure from
 \href{https://ulyngs.github.io/rmarkdown-workshop-2019/}{this R
 Markdown workshop}.

 Congratulations for taking a step further into the lands of open,
 reproducible science by writing your thesis using a tool that allows
 you to transparently include tables and dynamically generated plots
 directly from the underlying data. Hip hooray!
\end{abstract}
\end{center}

\thispagestyle{firstpages}
\end{titlepage}
\restoregeometry % Reset page geometry
\setcounter{page}{1} % Reset page counter
%%%%%%%%%%%%%%%%%
% END TITLE PAGES
%%%%%%%%%%%%%%%%%

%%%%%%%%%%%%%%%%%
% START MAIN CONTENT
%%%%%%%%%%%%%%%%%
% Start roman pages
\begin{romanpages}

% If you'd like to add a dedication, un-comment the following.
\begin{dedication}
  For X, if you wish
\end{dedication}

% Same for acknowledgements
\begin{acknowledgements}
 	This is where you will normally thank your advisor, colleagues, family
  and friends, as well as funding and institutional support. In our
  case, we will give our praises to the people who developed the ideas
  and tools that allow us to push open science a little step forward by
  writing plain-text, transparent, and reproducible theses in R
  Markdown.

  We must be grateful to John Gruber for inventing the original version
  of Markdown, to John MacFarlane for creating Pandoc
  (\url{http://pandoc.org}) which converts Markdown to a large number of
  output formats, and to Yihui Xie for creating \texttt{knitr} which
  introduced R Markdown as a way of embedding code in Markdown
  documents, and \texttt{bookdown} which added tools for technical and
  longer-form writing.

  Special thanks to \href{http://chester.rbind.io}{Chester Ismay}, who
  created the \texttt{thesisdown} package that helped many a PhD student
  write their theses in R Markdown. And a very special tahnks to John
  McManigle, whose adaption of Sam Evans' adaptation of Keith Gillow's
  original maths template for writing an Oxford University DPhil thesis
  in \LaTeX~provided the template that I adapted for R Markdown.

  Finally, profuse thanks to JJ Allaire, the founder and CEO of
  \href{http://rstudio.com}{RStudio}, and Hadley Wickham, the mastermind
  of the tidyverse without whom we'd all just given up and done data
  science in Python instead. Thanks for making data science easier, more
  accessible, and more fun for us all.

  \begin{flushright}
  Ulrik Lyngs \\
  Linacre College, Oxford \\
  2 December 2018
  \end{flushright}
\end{acknowledgements}

% Uncomment to generate a list of tables

% Uncomment to generate a list of abbreviations
% This might be a good place for a glossary, etc.
% First parameter can be changed eg to "Glossary" or something.
% Second parameter is the max length of bold terms.
\begin{mclistof}{List of Abbreviations}{3.2cm}

\item[1-D, 2-D] One- or two-dimensional, referring in this thesis to spatial dimensions in an image.

\item[Otter] One of the finest of water mammals.

\item[Hedgehog] Quite a nice prickly friend.

\end{mclistof} 


% end roman pages
\end{romanpages}

%%%%%%%%%%%%%%%%%
% CHAPTERS
%%%%%%%%%%%%%%%%%
% Add or remove any chapters you'd like here, by file name (excluding '.tex'):
\flushbottom

% all your chapters and appendices will appear here
\hypertarget{introduction}{%
\section*{Introduction}\label{introduction}}
\addcontentsline{toc}{section}{Introduction}

Welcome to the \emph{R Markdown} template for writing a PhD Dissertation
at the Graduate Institute of International and Development Studies in
Geneva. This sample content is based on the
\href{https://github.com/jhollway/iheidmytex}{IHEID LaTeX dissertation
template} and the R Bookdown package.

This project was inspired by by the
\href{https://github.com/ulyngs/oxforddown}{oxforddown},\href{https://github.com/ismayc/thesisdown}{thesisdown},
\href{https://github.com/benmarwick/huskydown}{huskydown} and
\href{https://github.com/rstudio/bookdown}{bookdown} packages. If you
are new to working with \texttt{bookdown} and \texttt{rmarkdown}, please
read over the great documentation provided by \texttt{thesisdown},
\texttt{oxforddown} and in the
\href{https://bookdown.org/yihui/bookdown/}{bookdown book}.


%%%%%%%%%%%%%%%%%
% BIBLIOGRAPHY
%%%%%%%%%%%%%%%%%
\setlength{\baselineskip}{0pt} % Single-space References

{\renewcommand*\MakeUppercase[1]{#1}%
\printbibliography[heading=bibintoc,title={\bibtitle}]}

%%%%%%%%%%%%%%%%% 
% END DOCUMENT BODY
%%%%%%%%%%%%%%%%%
\end{document}