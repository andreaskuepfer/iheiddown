%%%%%%%%%%%%%%%%%%%%%%%%%%%%%%%%%%%%%%%%%%%%%%%%%%%%%%%%%%%%%%%
%% IHEID THESIS TEMPLATE

% Use this template to produce a standard thesis that meets the Graduate Institute requirements for submission
%
% The template builds off of Keith A. Gillow's "ociamthesis.cls" LaTeX class (gillow@maths.ox.ac.uk, 1997).
% It also builds off of Ulrik Lyngs's "oxforddown" adaptation for RMarkdown (ulrik.lyngs@cs.ox.ac.uk, 2018)

%%%%% CHOOSE PAGE LAYOUT
\documentclass[a4paper, $if(page-layout)$$page-layout$$endif$]{templates/iheidthesis}

% UL 30 Nov 2018 pandoc puts lists in 'tightlist' command when no space between bullet points in Rmd file
\providecommand{\tightlist}{%
  \setlength{\itemsep}{0pt}\setlength{\parskip}{0pt}}
 
% UL 1 Dec 2018, fix to include code in shaded environments
$if(highlighting-macros)$
$highlighting-macros$
%UL 2 Dec 2018 add a bit of white space before and after code blocks
\renewenvironment{Shaded}
{
  \vspace{4pt}%
  \begin{snugshade}%
}{%
  \end{snugshade}%
  \vspace{4pt}%
}
$endif$

%UL 2 Dec 2018 reduce whitespace around verbatim environments
\usepackage{etoolbox}
\makeatletter
\preto{\@verbatim}{\topsep=0pt \partopsep=0pt }
\makeatother

%UL 26 Mar 2019, enable strikethrough
\usepackage[normalem]{ulem}

%UL 15 Oct 2019, enable link highlighting to be turned off from YAML
% \usepackage[colorlinks=false,pdfpagelabels,hidelinks=$hidelinks$]{hyperref}

%%%%% SELECT YOUR DRAFT OPTIONS
% Three options going on here; use in any combination.  But remember to turn the first two off before
% generating a PDF to send to the printer!

% This adds a "DRAFT" footer to every normal page.  (The first page of each chapter is not a "normal" page.)
% $if(draft)$
% \fancyfoot[C]{\emph{DRAFT Printed on \today}}
% $endif$

% This highlights (in blue) corrections marked with (for words) \mccorrect{blah} or (for whole
% paragraphs) \begin{mccorrection} . . . \end{mccorrection}.  This can be useful for sending a PDF of
% your corrected thesis to your examiners for review.  Turn it off, and the blue disappears.
% $if(params.corrections)$
% \correctionstrue
% $endif$

%%%%% BIBLIOGRAPHY SETUP
% Note that your bibliography will require some tweaking depending on your department, preferred format, etc.
% The options included below are just very basic "sciencey" and "humanitiesey" options to get started.
% If you've not used LaTeX before, I recommend reading a little about biblatex/biber and getting started with it.
% If you're already a LaTeX pro and are used to natbib or something, modify as necessary.
% Either way, you'll have to choose and configure an appropriate bibliography format...

% The science-type option: numerical in-text citation with references in order of appearance.
% \usepackage[style=numeric-comp, sorting=none, backend=biber, doi=false, isbn=false]{biblatex}
% \newcommand*{\bibtitle}{References}

% The humanities-type option: author-year in-text citation with an alphabetical works cited.
% \usepackage[style=authoryear, sorting=nyt, backend=biber, maxcitenames=2, useprefix, doi=false, isbn=false]{biblatex}
% \newcommand*{\bibtitle}{Works Cited}

%UL 3 Dec 2018: set this from YAML in index.Rmd
$if(bib-humanities)$
\usepackage[style=authoryear, sorting=nyt, backend=biber, maxcitenames=2, useprefix, doi=$if(doi-in-bibliography)$$doi-in-bibliography$$else$false$endif$, isbn=false, uniquename=false]{biblatex}
\newcommand*{\bibtitle}{$bibliography-heading-in-pdf$}
$else$
\usepackage[style=numeric-comp, sorting=none, backend=biber, doi=$doi-in-bibliography$, isbn=false]{biblatex}
\newcommand*{\bibtitle}{$bibliography-heading-in-pdf$}
$endif$

% This makes the bibliography left-aligned (not 'justified') and slightly smaller font.
\renewcommand*{\bibfont}{\raggedright\small}

% Change this to the name of your .bib file (usually exported from a citation manager like Zotero or EndNote).
\addbibresource{$bibliography$}


% Uncomment this if you want equation numbers per section (2.3.12), instead of per chapter (2.18):
%\numberwithin{equation}{subsection}


%%%%% THESIS / TITLE PAGE INFORMATION
% Everybody needs to complete the following:
\title{$title$}
\author{$author$}
% \department{$department$}

% Master's candidates who require the alternate title page (with candidate number and word count)
% must also un-comment and complete the following three lines:
%\masterssubmissiontrue
% \candidateno{$thesisno$}
% \wordcount{28,815}

% Uncomment the following line if your degree also includes exams (eg most masters):
%\renewcommand{\submittedtext}{Submitted in partial completion of the}
% Your full degree name.  (But remember that DPhils aren't "in" anything.  They're just DPhils.)
% \degree{$degree$}
% Term and year of submission, or date if your board requires (eg most masters)
% \degreedate{$degreedate$}


%%%%% YOUR OWN PERSONAL MACROS
% This is a good place to dump your own LaTeX macros as they come up.

% To make text superscripts shortcuts
	% \renewcommand{\st}{\textsuperscript{st}}
	% \newcommand{\nd}{\textsuperscript{nd}}
	% \newcommand{\rd}{\textsuperscript{rd}}
	% \renewcommand{\th}{\textsuperscript{th}} % ex: I won 4\th place

%%%%% THE ACTUAL DOCUMENT STARTS HERE
\begin{document}

%%%%% CHOOSE YOUR LINE SPACING HERE
% This is the official option.  Use it for your submission copy and library copy:
% \setlength{\textbaselineskip}{22pt plus2pt}
% This is closer spacing (about 1.5-spaced) that you might prefer for your personal copies:
%\setlength{\textbaselineskip}{18pt plus2pt minus1pt}

% You can set the spacing here for the roman-numbered pages (acknowledgements, table of contents, etc.)
% \setlength{\frontmatterbaselineskip}{17pt plus1pt minus1pt}

% UL: You can set the line and paragraph spacing here for the separate abstract page to be handed in to Examination schools
% \setlength{\abstractseparatelineskip}{13pt plus1pt minus1pt}
% \setlength{\abstractseparateparskip}{0pt plus 1pt}

% UL: You can set the general paragraph spacing here - I've set it to 2pt (was 0) so
% it's less claustrophobic
% \setlength{\parskip}{2pt plus 1pt}


% Leave this line alone; it gets things started for the real document.
% \setlength{\baselineskip}{\textbaselineskip}


%%%%% CHOOSE YOUR SECTION NUMBERING DEPTH HERE
% You have two choices.  First, how far down are sections numbered?  (Below that, they're named but
% don't get numbers.)  Second, what level of section appears in the table of contents?  These don't have
% to match: you can have numbered sections that don't show up in the ToC, or unnumbered sections that
% do.  Throughout, 0 = chapter; 1 = section; 2 = subsection; 3 = subsubsection, 4 = paragraph...

% The level that gets a number:
\setcounter{secnumdepth}{2}
% The level that shows up in the ToC:
\setcounter{tocdepth}{$toc-depth$}


%%%%% ABSTRACT SEPARATE
% This is used to create the separate, one-page abstract that you are required to hand into the Exam
% Schools.  You can comment it out to generate a PDF for printing or whatnot.
% $if(abstractseparate)$
% \begin{abstractseparate}
%   $abstract$
% \end{abstractseparate}
% $endif$

% JEM: Pages are roman numbered from here, though page numbers are invisible until ToC.  This is in
% keeping with most typesetting conventions.
% \begin{romanpages}

% Title page is created here
\maketitle

%%%%% DEDICATION -- If you'd like one, un-comment the following.
% $if(dedication)$
% \begin{dedication}
%   $dedication$
% \end{dedication}
% $endif$

%%%%% ACKNOWLEDGEMENTS -- Nothing to do here except comment out if you don't want it.
% $if(acknowledgements)$
% \begin{acknowledgements}
%  	$acknowledgements$
% \end{acknowledgements}
% $endif$


%%%%% ABSTRACT -- Nothing to do here except comment out if you don't want it.
\begin{abstract}
	$abstract$
\end{abstract}

%%%%% MINI TABLES
% This lays the groundwork for per-chapter, mini tables of contents.  Comment the following line
% (and remove \minitoc from the chapter files) if you don't want this.  Un-comment either of the
% next two lines if you want a per-chapter list of figures or tables.
% $if(mini-toc)$
%   \dominitoc % include a mini table of contents
% $endif$
% $if(mini-lof)$
%   \dominilof  % include a mini list of figures
% $endif$
% $if(mini-lot)$
%   \dominilot  % include a mini list of tables
% $endif$

% This aligns the bottom of the text of each page.  It generally makes things look better.
\flushbottom

% This is where the whole-document ToC appears:
\tableofcontents

% $if(lof)$
% \listoffigures
% 	\mtcaddchapter
%   	% \mtcaddchapter is needed when adding a non-chapter (but chapter-like) entity to avoid confusing minitoc
% $endif$
% 
% % Uncomment to generate a list of tables:
% $if(lot)$
% \listoftables
%   \mtcaddchapter
% $endif$
%%%%% LIST OF ABBREVIATIONS
% This example includes a list of abbreviations.  Look at text/abbreviations.tex to see how that file is
% formatted.  The template can handle any kind of list though, so this might be a good place for a
% glossary, etc.
$if(abbreviations)$
\include{$abbreviations$}
$endif$

% The Roman pages, like the Roman Empire, must come to its inevitable close.
% \end{romanpages}

%%%%% CHAPTERS
% Add or remove any chapters you'd like here, by file name (excluding '.tex'):
\flushbottom

% all your chapters and appendices will appear here
$body$


%%%%% REFERENCES

% JEM: Quote for the top of references (just like a chapter quote if you're using them).  Comment to skip.
% \begin{savequote}[8cm]
% The first kind of intellectual and artistic personality belongs to the hedgehogs, the second to the foxes \dots
%   \qauthor{--- Sir Isaiah Berlin \cite{berlin_hedgehog_2013}}
% \end{savequote}

\setlength{\baselineskip}{0pt} % JEM: Single-space References

{\renewcommand*\MakeUppercase[1]{#1}%
\printbibliography[heading=bibintoc,title={\bibtitle}]}

\end{document}
